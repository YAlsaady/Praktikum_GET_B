\section{Auswertung}

\subsection{Symmetrische Drehstromlast }
\begin{enumerate}[label=\alph*)]
  \item Berechnen Sie unter Verwendung von Gleichung (7) für die Belastungen nach 3.3 (a) die komplexe Scheinleistung $\underline S$ aus den komplexen Größen $\underline U_{1N}$, $\underline U_{2N}$, $\underline U_{3N}$ sowie $\underline I_{1}$, $\underline I_{2}$, $\underline I_{3}$, und ermitteln Sie hieraus die von der Last aufgenommene Wirk- und die Blindleistung.
    \begin{align*}
      \underline S &= \underline U_{1N} \cdot \underline I_1^* + 
                      +\underline U_{2N} \cdot \underline I_2^*
                      +\underline U_{3N} \cdot \underline I_3^*\\
      \underline S &= 232Ve^{j0^\circ}\cdot 0,96Ae^{-(-j70^\circ)} + 234Ve^{j240^\circ}\cdot 0,9Ae^{-(j170^\circ)} + 233Ve^{j120^\circ}\cdot 0,2Ae^{-(j50^\circ)}\\ 
      \underline S &= 650,44 VA \cdot e^{j70^\circ} = 223.82 W + j614.93 Var\\
      P &= Re\{\underline S\} = 223.82 W \\
      Q &= Im\{\underline S\} = 614.93 Var
      % \underline U_{12}&= \underline U_{1N} - \underline U_{2N} = 232Ve^{j0} - 234Ve^{j240} = 403,6Ve^{j30,1^\circ}\\
      % \underline U_{23}&= \underline U_{2N} - \underline U_{3N} = 234Ve^{j240} - 233Ve^{j120} = 404,4Ve^{-j90,1^\circ}\\
      % \underline U_{31}&= \underline U_{3N} - \underline U_{1N} = 233Ve^{j120} - 232Ve^{j0} = 401,8Ve^{149,9^\circ}\\
      % \underline U_m &= \frac{U_{12} + U_{23} + U_{31}}{3}\\
      % \underline U_m &= \frac{403,6V+404,4V+401,8V}{3}\\
      % \underline I_m &= \frac{\underline I_{1} + \underline I_{2} + \underline I_{1}}{3}\\
      % \underline I_m &= \frac{}{3}\\
      % \underline S &= \sqrt 3 \cdot \underline U_m \cdot \underline I_m\\
    \end{align*}
		
	\item Bestimmen Sie für 3.3 (a) den Wirkleistungsfaktor $\cos\varphi$ der Last. Verwenden Sie hierfür die Ergebnisse aus (a). 
    \begin{align*}
      \underline S &= |S|\cdot e^{j\varphi} = 650,44 VA \cdot e^{j70^\circ}\\
      \Rightarrow \varphi &= 70^\circ
    \end{align*}
	
  \item Zeichnen Sie für die Belastungen nach 3.3 (a) das Zeigerdiagramm der Spannungen $\underline{U}_{1N}$, $\underline U_{2N}$, $\underline U_{3N}$ und $\underline U_{K}$ sowie das Zeigerdiagramm der Ströme $\underline I_{1}$, $\underline I_{2}$, $\underline I_{3}$ und $\underline I_{N}$.
		 		
	 		\begin{figure}[h!]
	 			\centering
	 			\begin{subfigure}[b]{0.5\textwidth}
	 				\includegraphics
	 				[width=\textwidth]{img/img3.3.2.png}
	 			\end{subfigure}\hfil
	 			\caption{Skizze der Spannungen - Symmetrische Drehstromlast ohne Kompensation}
	 		\end{figure}
      \pagebreak
	 			 		
	 		\begin{figure}[h!]
	 			\centering
	 			\begin{subfigure}[b]{0.5\textwidth}
	 				\includegraphics
	 				[width=\textwidth]{img/img3.3.3.png}
	 			\end{subfigure}\hfil
	 			\caption{Skizze der Ströme - Symmetrische Drehstromlast ohne Kompensation}
	 		\end{figure}

	\item Bewerten Sie Ihre Beobachtungen für 3.3 (b), (d) und (e) anhand der aufgenommenen Oszillogramme und Ihrer Versuchsvorbereitungen.
	\\
	
	\textbf{In Abschnitt 3.3 (b)} haben wir eine symmetrische Verbindung mit dem Knotenpunkt untersucht und festgestellt, dass die Strangströme im Vergleich zur Spannung eine Phasenverschiebung von 70° aufweisen. Beim Anschluss des Neutralleiters an diesen Knotenpunkt konnten wir keinen signifikanten Unterschied feststellen, da die Last symmetrisch ist.
	\\ \ \\
	\textbf{Im Abschnitt 3.3 (d)} haben wir die Spannungen und Ströme bei der Dreieckkompensation analysiert. Dabei bemerkten wir, dass die Strangströme im Vergleich zur Spannung keine Phasenverschiebung aufwiesen. Allerdings konnten wir Ober- und Unterschwingungen in den Strömen beobachten.
	\\ \ \\
	\textbf{Im Abschnitt 3.3 (e)} betrachteten wir die Stern-Kompensation und stellten fest, dass die Strangströme im Vergleich zur Spannung eine Phasenverschiebung von 70° aufwiesen. Es besteht die Gefahr, dass die Kompensation nicht wie erwartet funktioniert, da die Kondensatoren dennoch etwa dreimal größer sein sollten, um die Ergebnisse zu erzielen.
\end{enumerate}

\subsection{Unsymmetrische Drehstromlast }
\begin{enumerate}[label=\alph*)]
	\item Vergleichen Sie die gemessenen Werten unter 3.4 (a) und (b) mit den in der Vorbereitung errechneten Messwerten. Bewerten Sie die Unterschiede.
	 			\begin{table}[h!]
          \begin{center}
            \caption{Vergleich die gemessenen Werten unter 3.4 (a) mit den errechneten Messwerten}
	 				\begin{tabular}{r | c c | c c}
	 					\hline
            &\multicolumn{2}{c |}{\textbf{Messung}} &
            \multicolumn{2}{c }{\textbf{Rechnung}} \\
	 					\hline
            Art & \(\underline U_{eff}\ in\ V \) & \( Phase\ \varphi\ in\ ^\circ\ \ \)&
	 					\(\underline U_{eff}\ in\ V \) & \( Phase\ \varphi\ in\ ^\circ\ \ \)\\
	 					\hline
            $\underline U_{1N}$ & \( 234 \) & \( 0   \)  & \( 220 \) & \( 0   \) \\
	 					$\underline U_{2N}$ & \( 233 \) & \( 240 \)  & \( 220 \) & \( 240 \) \\
            $\underline U_{3N}$ & \( 233 \) & \( 120 \)  & \( 220 \) & \( 120 \) \\
            $\underline U_{KN}$ & \( 0   \) & \( 0   \)  & \( 0   \) & \( 0   \) \\
	 					\hline
            Art & \(\underline I_{eff}\ in\ mA \) & \( Phase\ in\ ^\circ\ \ \) &
	 					\(\underline I_{eff}\ in\ mA \) & \( Phase\ in\ ^\circ\ \ \) \\
	 					\hline
            $\underline I_{1}$	 & \( 140 \) & \( -22 \)  & \( 140 \) & \( -17,77 \)\\
            $\underline I_{2}$	 & \( 116 \) & \( 240 \)  & \( 109 \) & \( 240 \)\\
	 					$\underline I_{3}$	 & \( 84  \) & \( 156 \)  & \( 81  \) & \( 156 \)\\
	 					$\underline I_{N}$	 & \( 114 \) & \( 90  \)  & \( 114 \) & \( 93  \)\\
	 					\hline
	 				\end{tabular}
	 		\end{center}
    \end{table}

      \pagebreak
      \begin{table}[h!]
        \begin{center}
	 				\caption{Vergleich die gemessenen Werten unter 3.4 (b) mit den errechneten Messwerten}
	 				\begin{tabular}{r | c c | c c}
	 					\hline
            &\multicolumn{2}{c |}{\textbf{Messung}} &
            \multicolumn{2}{c }{\textbf{Rechnung}} \\
	 					\hline
	 					Art & \(\underline U_{eff}\ in\ V \) & \( Phase\ \varphi\ in\ ^\circ\ \ \)  &
	 					\(\underline U_{eff}\ in\ V \) & \( Phase\ \varphi\ in\ ^\circ\ \ \)  \\
	 					\hline
	 					$\underline U_{1N}$ & \( 234 \) & \( 0   \) & \( 220 \) & \( 0   \) \\
	 					$\underline U_{2N}$ & \( 231 \) & \( 239 \) & \( 220 \) & \( 240 \) \\
            $\underline U_{3N}$ & \( 233 \) & \( 119 \) & \( 220 \) & \( 120 \) \\
            $\underline U_{KN}$ & \( 82  \) & \( 268 \) & \( 75  \) & \( 273 \) \\
	 					\hline
            Art & \(\underline I_{eff}\ in\ mA \) & \( Phase\ in\ ^\circ\ \ \)&
	 					\(\underline I_{eff}\ in\ mA \) & \( Phase\ in\ ^\circ\ \ \) \\
	 					\hline
            $\underline I_{1}$	 & \( 150 \) & \( 3   \) & \( 148 \) & \( 1,3   \) \\
	 					$\underline I_{2}$	 & \(  81 \) & \( 228 \) & \(  81 \) & \( 225 \) \\
	 					$\underline I_{3}$	 & \( 112 \) & \( 148 \) & \( 107 \) & \( 149 \) \\
	 					$\underline I_{N}$	 & \( 0   \) & \( 0   \) & \( 0   \) & \( 0   \) \\
	 					\hline
	 				\end{tabular}
	 		\end{center}
    \end{table}
	
	\item Berechnen Sie die komplexe Scheinleistung $\underline S$, und ermitteln Sie hieraus die Wirk- und die Blindleistung (dreiphasig) für die unter 3.4 gemessenen Drehstromlast
    \begin{align*}
      \underline S &= \underline U_{1N} \cdot \underline I_1^* + 
                      +\underline U_{2N} \cdot \underline I_2^*
                      +\underline U_{3N} \cdot \underline I_3^*\\
      \underline S &= 234Ve^{j0^\circ}\cdot 0,15Ae^{-(j3^\circ)} + 231Ve^{j239^\circ}\cdot 0,08Ae^{-(j228^\circ)} + 233Ve^{j119^\circ}\cdot 0,11Ae^{-(j148^\circ)}\\ 
      \underline S &=76,37 VA \cdot e^{-j8,08^\circ} = 75,61 W -j 10,74 Var\\
      P &= Re\{\underline S\} = 75,61 W \\
      Q &= Im\{\underline S\} = -10,74 Var
    \end{align*}

\end{enumerate}
