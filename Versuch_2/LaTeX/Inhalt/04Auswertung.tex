\section{Auswertung}

\subsection{Symmetrische Drehstromlast }
\begin{enumerate}[label=\alph*)]
	\item Berechnen Sie unter Verwendung von Gleichung (7) für die Belastungen nach 3.3 (a) die komplexe Scheinleistung $S$ aus den komplexen Größen $U_{1N}$, $U_{2N}$, $U_{3N}$ sowie $I_{1N}$, $I_{2N}$, $I_{3N}$, und ermitteln Sie hieraus die von der Last aufgenommene Wirk- und die Blindleistung.
	
	
	
	\item Bestimmen Sie für 3.3 (a) den Wirkleistungsfaktor $cos\phi$ der Last. Verwenden Sie hierfür die Ergebnisse aus (a). 
	
	\item Zeichnen Sie für die Belastungen nach 3.3 (a) das Zeigerdiagramm der Spannungen $U_{1N}$, $U_{2N}$, $U_{3N}$ und $U_{K}$ sowie das Zeigerdiagramm der Ströme $I_{1N}$, $I_{2N}$, $I_{3N}$ und $I_{N}$.
	
	\item Bewerten Sie Ihre Beobachtungen für 3.3 (b), (d) und (e) anhand der aufgenommenen Oszillogramme und Ihrer Versuchsvorbereitungen. 
\end{enumerate}

\subsection{Unsymmetrische Drehstromlast }
\begin{enumerate}[label=\alph*)]
	\item Vergleichen Sie die gemessenen Werten unter 3.4 (a) und (b) mit den in der Vorbereitung errechneten Messwerten. Bewerten Sie die Unterschiede.
	
	\item Berechnen Sie die komplexe Scheinleistung $S$, und ermitteln Sie hieraus die Wirk- und die Blindleistung (dreiphasig) für die unter 3.4 gemessenen Drehstromlast
\end{enumerate}
