\section{Messungen und Auswertung}

	\subsection{Welche Netzform hat das 400-V-Drehstromnetz?}
		\begin{enumerate}[label=\alph*)]
			\item 
		\end{enumerate}
		
	\subsection{Fehlerstromschutzschalter zum Personenschutz}
		\begin{enumerate}[label=\alph*)]
			\item Ermitteln Sie den tatsächlichen Auslösestrom des FI-Schutzschalters durch Mittelwertbildung aus drei Messungen, benutzen Sie dazu Multimeter und den FI-Tester aus Bild 9. Nutzen Sie dazu die Min/Max-Funktion des Multimeters. Stellen Sie das Potentiometer auf einen großen Wert ein und verkleinern Sie dann langsam den Widerstandswert. Der Auslösestrom kann dann am Multimeter abgelesen werden.
			
			\item Bestimmen Sie die Abschaltzeit bei Nennfehlerstrom des FI-Schutzschalters durch Mittelwertbildung aus drei Messungen, benutzen Sie dazu Multimeter, FI-Tester, Oszilloskop und Trennteiler. Ein Oszillogramm ist für die Ausarbeitung aufzubereiten! 
			
		\end{enumerate}
 	
 	\subsection{Symmetrische Drehstromlast ohne/mit Kompensation}
	 	\begin{enumerate}[label=\alph*)]
	 		\item Bestimmen Sie die komplexen Effektivwerte der Spannungen $U_{1N}$, $U_{2N}$, $U_{3N}$ und $U_{NK}$ und der Ströme $I_{1}$, $I_{2}$, $I_{3}$ und $I_{K}$ bei angeschlossener symmetrischer Drehstromlast nach Bild 12. Skizzieren Sie die Messschaltung und verwenden Sie zur Messung Oszilloskop, Trennteiler und Stromwandler. Das Oszillogramm von $u_{1}$ mit $i_{1}$ ist mit in die Ausarbeitung zu übernehmen und aufzuarbeiten.
	 		
	 		\item Verbinden Sie nun den Neutralleiter mit dem Knoten K der Sternschaltung und messen Sie den Strom IN. Nehmen Sie ein Oszillogramm auf und beschreiben Sie ihre Beobachtungen. Entfernen Sie anschließend wieder die Verbindung zwischen Neutralleiter und Sternpunkt K. 
	 		
	 		\item Bestimmen Sie aus den Messungen unter (a) die komplexe Impedanz $Z_{K}$ in kartesischen Koordinaten durch Mittelung der Werte $Z_{1K}$, $Z_{2K}$ und $Z_{3K}$. Ermitteln Sie anschließend die nötige Kompensationskapazität in Stern- und Dreieckschaltung. Nehmen Sie dazu an, dass die symmetrische Last aus entsprechenden Bauteilen in Reihenschaltung besteht.
	 		
	 		\item Beobachten Sie die Spannungen $u_{1N}$, $u_{2N}$, $u_{3N}$ und $u_{NK}$ die Ströme $i_{1}$, $i_{2}$, $i_{3}$ und $i_{N}$ bei angeschlossener symmetrischer Drehstromlast und drei Kondensatoren $C = 4 \mu F$ in Dreieckschaltung.  Skizzieren Sie die Schaltung und verwenden Sie zur Messung Oszilloskop, Trennteiler und Stromwandler. Halten Sie in einem Satz Ihre Beobachtungen mit einer entsprechenden Begründung fest (ein Oszillogramm in die Ausarbeitung übernehmen und aufarbeiten; $u_{1}$ mit $i_{1}$).
	 		
	 		\item Schalten Sie nun die drei Kondensatoren in Stern und schließen Sie diese jetzt parallel zur Last an. Überlegen Sie, welche Auswirkungen das auf den Phasenwinkel hat und begründen Sie Ihre Beobachtungen. 
	 	\end{enumerate}
	 	
 	\subsection{Unsymmetrische Drehstromlast}
	 	\begin{enumerate}[label=\alph*)]
	 		\item Bestimmen Sie die komplexen Effektivwerte der Spannungen $U_{1N}$, $U_{2N}$, $U_{3N}$ und $U_{NK}$ und der Ströme $I_{1}$, $I_{2}$, $I_{3}$ und $I_{K}$ bei der unsymmetrischen Drehstromlast nach Bild 13 \textcolor{red}{mit} angeschlossenem Neutralleiter. (Last je nach Gruppennummer)
	 		\newline
			Skizzieren Sie die Schaltung und verwenden Sie zur Messung Oszilloskop, Trennteiler und Stromwandler. Zwei Oszillogramme sind mit in die Ausarbeitung zu übernehmen und aufzuarbeiten. ($u_{1}$ mit $u_{KN}$, $u_{1}$ mit $i_{N}$).
			
			\item Bestimmen Sie die komplexen Effektivwerte der Spannungen $U_{1N}$, $U_{2N}$, $U_{3N}$ und $U_{NK}$ und der Ströme $I_{1}$, $I_{2}$, $I_{3}$ und $I_{K}$ bei der unsymmetrischen  Drehstromlast nach Bild 13 \textcolor{red}{ohne} angeschlossenem Neutralleiter. (Last je nach Gruppennummer)
			\newline
			Skizzieren Sie die Schaltung und verwenden Sie zur Messung Oszilloskop, Trennteiler und Stromwandler. Zwei Oszillogramme sind mit in die Ausarbeitung zu übernehmen und aufzuarbeiten. ($u_{1}$ mit $u_{KN}$, $u_{1}$ mit $i_{N}$). 
			
			
	 	\end{enumerate}