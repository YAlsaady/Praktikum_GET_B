\section{Vorbereitung}
 {Die Versuchsvorbereitung ist Bestandteil des Versuchs. Sie erhalten dafür ein gesondertes Testat.
  Ohne testierte Vorbereitung können Sie den Versuch nicht durchführen.}
\subsection{Ersatzschaltbild Transformator}
\begin{enumerate}[label=\alph*)]
  \item Zeichnen Sie das vollständige einphasige Ersatzschaltbild (Sternschaltung) des
        Transformators und geben Sie an, wie die einzelnen Werte und das
        Übersetzungsverhältnis ü aus dem Leerlauf- und dem Kurzschlussversuch berechnet
        werden können.
        \begin{figure}[h!]
          \begin{center}
            \includegraphics[width=0.95\textwidth]{img/2.1.1.1}
          \end{center}
          \caption{Vollständige einphasige Ersatzschaltbild des Transformators}\label{img:2.1.1.1}
        \end{figure}
        $$ü=\frac{U_1}{U_2}=\frac{w_1}{w_2}=\frac{I_2}{I_1}$$
        \begin{minipage}{0.5\textwidth}
          Leerlauf:
          \begin{align*}
            R'_2         & =R_2\cdot ü^2           \\
            U'_2         & =U_2\cdot ü             \\
            I'_2         & = \frac{I_2}{ü}         \\
            X'_{2\sigma} & = X_{2\sigma} \cdot ü^2 \\
            P'_2         & = P_2
          \end{align*}
        \end{minipage}\hfill
        \begin{minipage}{0.5\textwidth}
          Leerlauf:
          \begin{align*}
            R'_2         & =R_2\cdot ü^2           \\
            U'_2         & =U_2\cdot ü             \\
            I'_2         & = \frac{I_2}{ü}         \\
            X'_{2\sigma} & = X_{2\sigma} \cdot ü^2 \\
            P'_2         & = P_2
          \end{align*}
        \end{minipage}

        \pagebreak
  \item Machen Sie sich die Bedeutung der einzelnen Bauelemente des vollständigen
        Ersatzschaltbildes klar (wird abgefragt). Welche Bedeutung haben die
        „gestrichenen“ Größen?
        \begin{figure}[h!]
          \begin{center}
            \includegraphics[width=0.8\textwidth]{img/2.1.2.4.png}
          \end{center}
          \caption{Zeichnung der eines Realentransformators mit Schaltpan }\label{img/2.1.2.4}
        \end{figure}
        \\
        Das Ersatzschaltbild besteht aus verschiedenen Verlustkomponenten. Der Eisenkern bewirkt Energieverluste aufgrund von Hysterese und Wirbelströmen, die im Kern erzeugt werden. Sowohl an den Primär- als auch an den Sekundäranschlüssen befinden sich Kupferwicklungen, die aufgrund ihrer Länge ohmsche Verluste erzeugen, insbesondere unter Wechselstrombedingungen, wo Stromverdrängungseffekte auftreten. Der magnetische Fluss erstreckt sich nicht nur durch den Eisenkern, sondern breitet sich auch durch die Umgebungsluft aus, was zu einem Streufluss führt. Dieser Streufluss trägt nicht zur Energieübertragung bei.\\ \ \\
        \begin{figure}[h!]
          \begin{center}
            \includegraphics[width=0.8\textwidth]{img/2.1.2.2.png}
          \end{center}
          \caption{Schaltplan von einem realen Transformator mit einem idealen Transformator}\label{img/2.1.2.2}
        \end{figure}
        Die Striche an den Bauteilen signalisieren, dass diese bereits transformiert sind. Um die Spannung und den Strom umzuwandeln, muss das Ersatzschaltbild um einen idealen Transformator auf der Sekundärseite erweitert werden, da allein passive Bauelemente keine Transformationsfunktion erfüllen können.\\
        \begin{figure}[h!]
          \begin{center}
            \includegraphics[width=0.8\textwidth]{img/2.1.2.3.png}
          \end{center}
          \caption{Schaltplan von einem realen Transformator mit einem Lastwiderstand}\label{img/2.1.2.3}
        \end{figure}

  \item Wie groß darf der Strom im Kurzschlussversuch maximal werden? Begründen Sie
        Ihre Aussage.\\ \ \\

        Im Kurzschlussversuch bei einem Transformator wird die Eingangsspannung
        schrittweise erhöht, bis der Strom auf der Primärseite den Nennstrom $(I_{1N})$
        erreicht. Dabei muss beachtet werden, dass der Kurzschlussversuch eine Art von
        Überlasttest ist, und es ist wichtig, die zulässigen Grenzen einzuhalten, um
        Schäden am Transformator zu vermeiden.
        \[ I_K = I_{1N} \]
        Die maximale Strombelastung im Kurzschlussversuch wird durch die thermische
        Belastung des Transformators begrenzt. Wenn der Strom zu hoch wird, kann dies
        zu übermäßiger Erwärmung führen, was zu Isolationsversagen oder anderen Schäden
        führen kann. Daher sollte der Kurzschlusstest so durchgeführt werden, dass der
        Transformator nicht beschädigt wird.

        Der Strom im Kurzschlussversuch sollte normalerweise auf einen Wert begrenzt
        werden, der sicher unterhalb des Kurzschlussspitzenstroms $(I_{k})$ liegt. Der
        Kurzschlussspitzenstrom tritt aufgrund von Transienten auf, wenn der
        Transformator plötzlich kurzgeschlossen wird. Er kann ein Vielfaches des
        Nennstroms betragen und ist kurzzeitig.

        Die genaue maximale zulässige Strombelastung im Kurzschlussversuch hängt von
        den spezifischen Eigenschaften des Transformators ab, einschließlich seiner
        Größe, Bauart und Kühlung. Herstellerangaben und Normen sollten für eine genaue
        Bestimmung konsultiert werden. Es ist wichtig, den Kurzschlussversuch
        sorgfältig zu planen und sicherzustellen, dass er den Spezifikationen des
        Transformators entspricht, um Schäden zu vermeiden und die Sicherheit zu
        gewährleisten.

  \item Leiten Sie her, wie die Verlustleistung des Transformators in einem beliebigen
        Betriebspunkt (vorgegeben durch den Strom $I_1$) aus den Verlusten der
        Leerlauf- und Kurzschlussmessung ermittelt werden kann.
  \item Leiten Sie anhand des Zeigerdiagramms (Bild 6) die Formeln für den Längs- und
        Querspannungsabfall her.
        \begin{align*}
          \underline{\Delta U} & = \underline Z\cdot \underline I_1                                                                                                        \\
          \underline{\Delta U} & = (R_k+ jX_k)\cdot (I_1\cdot e^{-j\varphi_2})                                                                                             \\
          \underline{\Delta U} & = (R_k+ jX_k)\cdot \left(I_1\cdot\cos(-\varphi_2)+j\cdot I_1\cdot \sin(-\varphi_2)\right)                                                 \\
          \underline{\Delta U} & = (R_k+ jX_k)\cdot \left(I_1\cdot\cos\varphi_2-j\cdot I_1\cdot \sin\varphi_2\right)                                                       \\
          \underline{\Delta U} & = R_K\cdot I_1\cdot \cos\varphi_2 + X_K\cdot I_1 \cdot \sin\varphi_2+J(-R_K\cdot I_1 \cdot \sin\varphi_2+X_K\cdot I_1\cdot \cos\varphi_2) \\
          \underline{\Delta U} & = U_\ell +jU_q                                                                                                                            \\
          \Rightarrow U_\ell   & =Re{\underline{\Delta U}}=R_K\cdot I_1 \cdot\cos \varphi_2 + X_K \cdot I_1\cdot \sin \varphi_2                                            \\
          \Rightarrow U_q      & =Im{\underline{\Delta U}}= -R_K\cdot I_1 \cdot\sin \varphi_2 + X_K \cdot I_1\cdot \cos \varphi_2                                          \\
        \end{align*}
  \item Leiten Sie anhand des Zeigerdiagramms (Bild 6) und den Formeln für den Längs-
        und Querspannungsabfall die Formel (12) für den relativen Spannungsfall auf der
        Sekundärseite her.
        \begin{align*}
          \Delta U'_2                            & = \frac{U_1 - U'_2}{U_1}                                                                                                \\
          \Delta U'_2                            & = 1-\frac{U'_2}{U_1}                                                                                                    \\
          \underline U'_2+\underline U_\ell      & = \sqrt{\underline U_1^2-\underline U_q^2}                                                                              \\
          \underline U'_2                        & = \sqrt{\underline U_1^2-\underline U_q^2}-\underline U_\ell                                                            \\
          \frac{\underline U'_2}{\underline U_1} & = \frac{\sqrt{\underline U_1^2-\underline U_q^2}}{\underline U_1}-\displaystyle\frac{\underline U_\ell}{\underline U_1} \\
          \frac{\underline U'_2}{\underline U_1} & = {\sqrt{\frac{\underline U_1^2-\underline U_q^2}{\underline U_1^2}}}
          -\displaystyle\frac{\underline U_\ell}{\underline U_1}                                                                                                           \\
          \frac{\underline U'_2}{\underline U_1} & = {\sqrt{1-\left(\frac{\underline U_q}{\underline U_1}\right)^2}}
          -\displaystyle\frac{\underline U_\ell}{\underline U_1}                                                                                                           \\
          \Delta U'_2                            & = 1-\frac{U'_2}{U_1}                                                                                                    \\
          \Delta U'_2                            & = 1 + \frac{U_\ell}{U_1}- \sqrt{1-\left(\frac{U_q}{U_1}\right)^2}
        \end{align*}
  \item Welche Aussagekraft hat die relative Kurzschlussspannung für den Betrieb des
        Transformators?

        Die relative Kurzschlussspannung eines Transformators ist ein entscheidender
        Parameter, der seine Verhaltensweise unter Belastung widerspiegelt. Dieser Wert
        ist von großer Bedeutung bei der Auslegung von Niederspannungsverteilungen
        sowie bei der Dimensionierung von Komponenten wie Sicherungsleisten und
        Leistungsschaltern.

        Die relative Kurzschlussspannung gibt an, wie stark die Spannung ansteigt, wenn
        die Sekundärseite des Transformators kurzgeschlossen wird und der Nennstrom auf
        der Sekundärseite fließt. Dies ist von Bedeutung, da im Falle eines
        Kurzschlusses in der Verteilung die Sekundärseite des Transformators weiterhin
        am Netz angeschlossen ist, bis der vorgeschaltete Schutzmechanismus auslöst.

        Die Kenntnis der relativen Kurzschlussspannung ermöglicht die Berechnung des
        maximalen Kurzschlussstroms, der in diesem Szenario auftreten kann. Ein
        einfacher Dreisatz kann verwendet werden, um dies zu ermitteln. Als Beispiel
        kann ein Drehstromtransformator mit den Spezifikationen $6\ KV / 400\ V$ und
        einer relativen Kurzschlussspannung von $6\ \%$ dienen. Der sekundäre Nennstrom
        beträgt $145\ A$.

        Die Berechnung des maximalen Kurzschlussstroms erfolgt durch die Division des
        Nennstroms durch die relative Kurzschlussspannung in Prozent. In diesem Fall
        ergibt sich ein Wert von $24\ A$. Dieser Strom, hier $2400\ A$, kann kurzzeitig
        durch die nach dem Transformator geschalteten Schutzschalter fließen. Es ist
        jedoch zu beachten, dass die dynamischen Kräfte, die bei einem Kurzschluss
        auftreten, die Schaltelemente mechanisch belasten und sie gegebenenfalls
        beschädigen können.

        Es ist daher entscheidend, die Kurzschlussfestigkeit der verwendeten
        Komponenten zu überprüfen und gegebenenfalls hochwertigere Komponenten
        auszuwählen. In vielen Fällen sind gängige Niederspannungsschaltgeräte jedoch
        für Kurzschlussströme von $6\ kA$ ausgelegt und erfüllen die Anforderungen.
\end{enumerate}

\subsection{Aronschaltung zur Leistungsmessung }
\begin{enumerate}[label=\alph*)]
  \item Leiten Sie her, warum Sie mit der Aronschaltung (Bild 8) die gesamte
        Scheinleistung im Dreiphasensystem messen können, sofern der Summenstrom im
        Knoten K zu Null angenommen werden kann.

        \begin{figure}[!h]
          \begin{minipage}[ct]{.5\linewidth}
            \begin{center}
              \includegraphics[width=\linewidth]{img/2.2.1.1.png}
            \end{center}
            \caption{Schaltplan von einem realen Transformator mit einem Lastwiderstand}\label{img/2.2.1.1}
          \end{minipage}
          \hspace{.1\linewidth}
          \begin{minipage}[ct]{.4\linewidth}
            \begin{center}
              \includegraphics[width=\linewidth]{img/2.2.1.2.png}
            \end{center}
            \caption{Zeigerdiagram einer Sternschaltung}\label{img/2.2.1.2}
          \end{minipage}
        \end{figure}

        \begin{align*}
          \text{Kontenregel gilt: ges.:}\  \underline{i}_3                                                                                                                 \\
          0               & =\ \underline{i}_1 + \underline{i}_2 + \underline{i}_3                                                                                         \\
          \underline{i}_3 & = - \underline{i}_1 - \underline{i}_2                                                                                                          \\
          \text{Leistungsgleichung:}                                                                                                                                       \\
          \underline{S}   & =\ \underline{U} \cdot\ \underline{I}\ =\ \underline{u} \cdot \underline{i}                                                                    \\
          \underline{S}   & =\ \underline{u}_1 \cdot \underline{i}_1 + \underline{u}_2 \cdot \underline{i}_2 + \underline{u}_3 \cdot \underline{i}_3                       \\
          \underline{S}   & =\ \underline{u}_1 \cdot \underline{i}_1 + \underline{u}_2 \cdot \underline{i}_2 + \underline{u}_3 \cdot (- \underline{i}_1 - \underline{i}_2) \\
          %S &= \ \underline{u}_1 \cdot \underline{i}_1 + \underline{u}_2 \cdot \underline{i}_2 - \underline{u}_3 \cdot \underline{i}_1 - \underline{u}_3 \cdot \underline{i}_2)\\
          \underline{S}   & = \underline{i}_1(\underline{u}_1 - \underline{u}_3) + \underline{i}_2(\underline{u}_2 - \underline{u}_3)                                      \\
          \underline{S}   & = \underline{i}_1\cdot\underline{u}_{13} + \underline{i}_2\cdot\underline{u}_{23}
        \end{align*}
        \begin{align*}
          \text{Phasenverschiebungswinkel:}                                          \\
          L_{M}\ =\ \text{Leistungsmessgerät}                                        \\ \ \\
          +\ L_{M1}   & =\ +\ L_{M2}\ \ \ \ \{ \varphi =\ 0^\circ                    \\
          +\ L_{M1}   & =\ -\ L_{M2}\ \ \ \ \{ \varphi =\ \pm90^\circ                \\
          \ L_{M1}    & \neq\ \ L_{M2}\ \ \ \ \{ 0^\circ< \varphi < 60^\circ         \\
          -\ L_{M1}\  & \text{oder}\ -\ L_{M2}\ \ \ \ \{ 60^\circ< \varphi < 0^\circ \\
        \end{align*}
  \item Skizzieren Sie das Ersatzschaltbild für den Transformator im Leerlauffall.
        Zeichnen Sie das zugehörige qualitative Zeigerdiagramm für den Leerlauffall.
        \begin{figure}[h!]
          \begin{center}
            \includegraphics[width=0.95\textwidth]{img/2.2.2.1}
          \end{center}
          \caption{ESP für den Transformator im Leerlauffall}\label{img:2.2.2.1}
        \end{figure}

  \item Zeigen Sie anhand dieses Zeigerdiagramms, warum Sie im Leerlaufversuch (Bild 9)
        mit der Aronschaltung eine negative Wirkleistung P1 messen.
\end{enumerate}

\subsection{Symmetrische Drehstromlast }
\begin{enumerate}[label=\alph*)]
  \item Geben Sie an, wie Sie aus den Messwerten nach 3.2 (a) und (b) die gemittelten
        Größen $U_m,\ I_m$ (siehe Versuch E2-4), den Leistungsfaktor
        $\lambda=\cos(\varphi)$, die Leistungen $S,\ P\ und\ Q$ jeweils für die Primär-
        und die Sekundärseite berechnen können.\\ \ \\ Leistungsfaktor $\lambda(\cos
            (\varphi))$:
        \[ \lambda = \frac{|P|}{S}=\frac{S\cdot |\cos \varphi|}{S} = |\cos \varphi| \]
        Der Leistungsfaktor kann zwischen $0$ und $1$ liegen: $0 \leq \lambda \leq 1
        $\\ Leistungen $S$, $P$ und $Q$:\\
        \begin{align*}
          P & = U_M\cdot I_M\cdot \lambda \\
          S & = U_M\cdot I_M              \\
          Q & = \sqrt{S^2-P^2}
        \end{align*}

  \item Mit welcher Leistung können Sie den Transformator ($Yy0$) maximal belasten?
\end{enumerate}
