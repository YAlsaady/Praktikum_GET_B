\section{Vorbereitung}
{Die Versuchsvorbereitung ist Bestandteil des Versuchs. Sie erhalten dafür ein gesondertes Testat.
Ohne testierte Vorbereitung können Sie den Versuch nicht durchführen.}
\subsection{Ersatzschaltbild Transformator}
\begin{enumerate}[label=\alph*)]
  \item Zeichnen Sie das vollständige einphasige Ersatzschaltbild (Sternschaltung) des Transformators und geben Sie an, wie die einzelnen Werte und das Übersetzungsverhältnis ü aus dem Leerlauf- und dem Kurzschlussversuch berechnet werden können. 
  \item Machen Sie sich die Bedeutung der einzelnen Bauelemente des vollständigen Ersatzschaltbildes klar (wird abgefragt). Welche Bedeutung haben die „gestrichenen“ Größen? 
  \item Wie groß darf der Strom im Kurzschlussversuch maximal werden? Begründen Sie Ihre Aussage. 
  \item Leiten Sie her, wie die Verlustleistung des Transformators in einem beliebigen Betriebspunkt (vorgegeben durch den Strom $I_1$) aus den Verlusten der Leerlauf- und Kurzschlussmessung ermittelt werden kann.  
  \item Leiten Sie anhand des Zeigerdiagramms (Bild 6) die Formeln für den Längs- und Querspannungsabfall her.  
  \item Leiten Sie anhand des Zeigerdiagramms (Bild 6) und den Formeln für den Längs- und Querspannungsabfall die Formel (12) für den relativen Spannungsfall auf der Sekundärseite her. 
  \item Welche Aussagekraft hat die relative Kurzschlussspannung für den Betrieb des Transformators? 
\end{enumerate}
\subsection{Aronschaltung zur Leistungsmessung }
\begin{enumerate}[label=\alph*)]
  \item Leiten Sie her, warum Sie mit der Aronschaltung (Bild 8) die gesamte Scheinleistung im Dreiphasensystem messen können, sofern der Summenstrom im Knoten K zu Null angenommen werden kann.  
  \item Skizzieren Sie das Ersatzschaltbild für den Transformator im Leerlauffall. Zeichnen Sie das zugehörige qualitative Zeigerdiagramm für den Leerlauffall.
  \item Zeigen Sie anhand dieses Zeigerdiagramms, warum Sie im Leerlaufversuch (Bild 9) mit der Aronschaltung eine negative Wirkleistung P1 messen. 
\end{enumerate}
\subsection{Symmetrische Drehstromlast }
\begin{enumerate}[label=\alph*)]
  \item Geben Sie an, wie Sie aus den Messwerten nach 3.2 (a) und (b) die gemittelten Größen $U_m,\ I_m$ (siehe Versuch E2-4), den Leistungsfaktor $\lambda=\cos(\varphi)$, die Leistungen $S,\ P\ und\ Q$ jeweils für die Primär- und die Sekundärseite berechnen können. 
  \item Mit welcher Leistung können Sie den Transformator ($Yy0$) maximal belasten?
\end{enumerate}
