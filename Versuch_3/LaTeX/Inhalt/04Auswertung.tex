\section{Auswertung}
\subsection{Ersatzschaltbild Transformator}
\begin{enumerate}[label=\alph*)]
	\item Zeichnen Sie das vollständige einphasige Ersatzschaltbild (Sternschaltung) des
	      Transformators.
	      \begin{figure}[h!]
		      \begin{center}
			      \includegraphics[width=0.95\textwidth]{img/4.1.1.1}
		      \end{center}
		      \caption{Vollständige einphasige ESP des Transformators}\label{img:2.1.1.1}
	      \end{figure}

	\item Bestimmen Sie die Leistung im Leerlauf auf der Primärseite nach Gleichung (16).
	      \begin{align*}
		      \underline S   & = \underline U_{12} \cdot \underline I_1^* + \underline U_{32}\cdot \underline I_3^* \\
		      \underline S   & = \underline U_{12} \cdot \underline I_1^* - \underline U_{23}\cdot \underline I_3^* \\
		      \underline S_0 & = 395\ V \cdot e^{j0^\circ} \cdot 0.45\ A \cdot e^{-(-j120^\circ)} -
		      393\ V \cdot e^{-j110^\circ}\cdot 0.45\ A \cdot e^{-(j25^\circ)}                                      \\
		      \underline S_0 & = 36,2\ W + j279\ \text{var} = 281\ VA\cdot e^{j82,6^\circ}                          \\
		      P_0            & = 36,2\ W                                                                            \\
		      Q_0            & = 279\ \text{var}
	      \end{align*}

	\item Bestimmen Sie die Wirkleistung im Kurzschlussfall auf der Sekundärseite nach
	      Gleichung (16). Gehen Sie dabei von einem symmetrischen System aus.
	      \begin{align*}
		      \underline S   & = \underline U_{12} \cdot \underline I_1^* + \underline U_{32}\cdot \underline I_3^* \\
		      \underline S   & = \underline U_{12} \cdot \underline I_1^* - \underline U_{23}\cdot \underline I_3^* \\
		      \underline S_K & = 12.5\ V \cdot e^{j0^\circ} \cdot 7,3\ A \cdot e^{-(-j52^\circ)}
		      - 12\ V \cdot e^{-j121^\circ}\cdot 7\ A \cdot e^{-(j65^\circ)}                                        \\
		      \underline S_K & = 140\ W+j 63\ \text{var} = 153\ VA \cdot e^{j24^\circ}                              \\
		      P_K            & = 140\ W                                                                             \\
		      Q_K            & = 63\ \text{var}
	      \end{align*}

	\item Ermitteln Sie die Daten des vollständigen Ersatzschaltbildes des Transformators
	      mit der Annahme $X_{1\sigma} = X'_{2\sigma} \text{ sowie } R_1 = R'_2$.
	      \begin{align*}
		      R_1                     & = R'_2                                                             \\
		      \Rightarrow R_1         & =\frac{P_{1K}}{2I_1^2}                                             \\
		      R_1                     & = \frac{140\ W}{3}\cdot \frac{1}{2(7,4\ A)^2}                      \\
		      R_1                     & = R'_2= 0,43\ \Omega                                               \\
		      X_{1\sigma}             & = X'_{2\sigma}                                                     \\
		      %
		      \Rightarrow X_{1\sigma} & =\frac{Q_{1K}}{2I_1^2}                                             \\
		      X_{1\sigma}             & = \frac{63\ \text{var}}{3}\cdot \frac{1}{2(7,4\ A)^2}              \\
		      X_{1\sigma}             & = X'_{2\sigma}= 0,19\ \Omega                                       \\
		      %
		      R_{FE}                  & = \frac{3\cdot(U_1)^2}{P_0}=\frac{3\cdot(U_{12})^2}{3 \cdot P_0} =
		      \frac{(U_{12})^2}{P_0}                                                                       \\ R_{FE} & = \frac{(395\ V)^2}{36,2\ W} \\ R_{FE} & =
		      4,3\ k\Omega                                                                                 \\
		      %
		      X_{h}                   & = \frac{3\cdot(U_1)^2}{Q_0}=\frac{3\cdot(U_{12})^2}{3 \cdot Q_0} =
		      \frac{(U_{12})^2}{Q_0}                                                                       \\ X_{h} & = \frac{(395\ V)^2}{279\ \text{var}} \\ X_{h}
		                              & = 559,2\ k\Omega
	      \end{align*}

	\item Berechnen Sie anhand des Kurzschlussversuches die relative Kurzschlussspannung
	      des Transformators.
	      \begin{align*}
		      u_k & = \frac{U_K}{U_N}        \\
		      u_k & = \frac{12\ V}{400\ V}   \\
		      u_k & = 0,03\ \widehat{=}\ 3\%
	      \end{align*}

	      \pagebreak

\end{enumerate}

\subsection{Symmetrische Drehstromlast}
\begin{enumerate}[label=\alph*)]

	\item Berechnen Sie aus den Messwerten nach 3.2(b) die gemittelten Größen $U_m, I_m
		      \text{ und } \varphi_m$ (siehe Versuch E2-4), den Leistungsfaktor $\lambda =
		      \cos(\varphi_m)$, die Leistungen $S, P \text{ und } Q$ jeweils für die Primär-
	      und die Sekundärseite.

	      \begin{figure}[h!]
		      \begin{center}
			      \includegraphics[width=0.95\textwidth]{img/4.2.1.1}
		      \end{center}
		      \caption{ESB mit Messwerten aus der Messung 3.2 b}\label{img:4.2.1.1}
	      \end{figure}
	      \textbf{Spannung}:\\\ \\
	      \begin{tcolorbox}[colback=gray!30,
			      colframe=black,
			      width=0.9\textwidth ]
		      \parbox{\textwidth}{

			      \begin{minipage}{0.5\textwidth}
				      \textbf{Primärseite:}\\ \ \\
				      \begin{align*}
					      U_{M.pri} & = \frac{U_{12} + U_{23} + U_{31}}{3} \\
					      U_{M.pri} & = \frac{378\ V + 371\ V + 372\ V}{3} \\
					      U_{M.pri} & = 373,67\ V
				      \end{align*}
			      \end{minipage}\hfill
			      \begin{minipage}{0.5\textwidth}
				      \textbf{Sekundärseite:}\\ \ \\
				      \begin{align*}
					      U_{M.sec} & = \frac{U_{12} + U_{23} + U_{31}}{3} \\
					      U_{M.sec} & = \frac{378\ V + 371\ V + 358\ V}{3} \\
					      U_{M.sec} & = 354\ V
				      \end{align*}
			      \end{minipage}
		      }
	      \end{tcolorbox}

	      \textbf{Strom}:\\\ \\
	      \begin{tcolorbox}[colback=gray!30,
			      colframe=black,
			      width=0.9\textwidth,
		      ]
		      \parbox{\textwidth}{

			      \begin{minipage}{0.5\textwidth}
				      \textbf{Primärseite:}\\ \ \\
				      \begin{align*}
					      I_{M.pri} & = \frac{I_{1} + I_{2} + I_{3}}{3} \\
					      I_{M.pri} & = \frac{7\ A + 7\ A + 7\ A}{3}    \\
					      I_{M.pri} & = 7\ A
				      \end{align*}
			      \end{minipage}\hfill
			      \begin{minipage}{0.5\textwidth}
				      \textbf{Sekundärseite:}\\ \ \\
				      \begin{align*}
					      I_{M.sec} & = \frac{I_{12} + I_{23} + I_{31}}{3} \\
					      I_{M.sec} & = \frac{7\ A + 7\ A + 6,7\ A}{3}     \\
					      I_{M.sec} & = 6,9\ A
				      \end{align*}
			      \end{minipage}
		      }
	      \end{tcolorbox}

	      \textbf{Phase}:\\\ \\
	      \begin{tcolorbox}[colback=gray!30,
			      colframe=black,
			      width=0.9\textwidth,
		      ]
		      \parbox{\textwidth}{

			      \begin{minipage}{0.5\textwidth}
				      \textbf{Primärseite:}\\ \ \\
				      \begin{align*}
					      \varphi_{M.pri} & = \frac{(\varphi_{1} + \varphi_{2} + \varphi_{3})-30^\circ}{3} \\
					      \varphi_{M.pri} & = \frac{(65^\circ+68^\circ+67^\circ)-30^\circ}{3}              \\
					      \varphi_{M.pri} & = 36,7^\circ                                                   \\
				      \end{align*}
			      \end{minipage}\hfill
			      \begin{minipage}{0.5\textwidth}
				      \textbf{Sekundärseite:}\\ \ \\
				      \begin{align*}
					      \varphi_{M.sec} & = \frac{(\varphi_{1} + \varphi_{2} + \varphi_{3})-30^\circ}{3} \\
					      \varphi_{M.sec} & = \frac{(64^\circ+65^\circ+63^\circ)-30^\circ}{3}              \\
					      \varphi_{M.sec} & = 34^\circ                                                     \\
				      \end{align*}
			      \end{minipage}
			      \\ \ \\
			      \centering
			      Minus $30^\circ$ kommen zustande, durch den Phasenwinkel der Außenleiterspannungen zu den Stangspannungen.
		      }
	      \end{tcolorbox}

	      \begin{figure}[h!]
		      \begin{center}
			      \includegraphics[width=0.95\textwidth]{img/4.2.1.2}
		      \end{center}
		      \caption{Zeigerdiagramm der Außenleiterspannungen und Strangspannungen}\label{img:4.2.1.2}
	      \end{figure}

	      \textbf{Scheinleistung:}\\ \ \\
	      \begin{tcolorbox}[colback=gray!30,
			      colframe=black,
			      width=0.9\textwidth,
		      ]
		      \parbox{\textwidth}{

			      \begin{minipage}{0.5\textwidth}
				      \textbf{Primärseite:}\\ \ \\
				      \begin{align*}
					      S_{pri} & = \sqrt3\cdot U_{M.pri}\cdot I_{M.pri} \\
					      S_{pri} & = \sqrt3\cdot 373,67\ V\cdot 7\ A      \\
					      S_{pri} & = 4,53\ \text{kVA}                     \\
				      \end{align*}
			      \end{minipage}\hfill
			      \begin{minipage}{0.5\textwidth}
				      \textbf{Sekundärseite:}\\ \ \\
				      \begin{align*}
					      S_{sec} & = \sqrt3\cdot U_{M.sec}\cdot I_{M.sec} \\
					      S_{sec} & = \sqrt3\cdot 354\ V\cdot 6,9\ A       \\
					      S_{sec} & = 4,23\ \text{kVA}                     \\
				      \end{align*}
			      \end{minipage}
		      }
	      \end{tcolorbox}
	      \textbf{Wirkleistung:}\\ \ \\
	      \begin{tcolorbox}[colback=gray!30,
			      colframe=black,
			      width=0.9\textwidth,
		      ]
		      \parbox{\textwidth}{

			      \begin{minipage}{0.5\textwidth}
				      \textbf{Primärseite:}\\ \ \\
				      \begin{align*}
					      P_{pri} & = S_{pri} \cdot \cos(\varphi_{M.pri})      \\
					      P_{pri} & = 4,53\ \text{kVA} \cdot \cos(36,67^\circ) \\
					      P_{pri} & = 3,63\ kW
				      \end{align*}
			      \end{minipage}\hfill
			      \begin{minipage}{0.5\textwidth}
				      \textbf{Sekundärseite:}\\ \ \\
				      \begin{align*}
					      P_{sec} & = S_{sec} \cdot \cos(\varphi_{M.sec})   \\
					      P_{sec} & = 4,23\ \text{kVA} \cdot \cos(34^\circ) \\
					      P_{sec} & = 3,51\ kW
				      \end{align*}
			      \end{minipage}
		      }
	      \end{tcolorbox}

	      \textbf{Blindleistung}\\ \ \\
	      \begin{tcolorbox}[colback=gray!30,
			      colframe=black,
			      width=0.9\textwidth,
		      ]
		      \parbox{\textwidth}{

			      \begin{minipage}{0.5\textwidth}
				      \textbf{Primärseite:}\\ \ \\
				      \begin{align*}
					      Q_{pri} & = S_{pri} \cdot \sin(\varphi_{M.pri})      \\
					      P_{pri} & = 4,53\ \text{kVA} \cdot \sin(36,67^\circ) \\
					      Q_{pri} & = 2,71\ \text{kvar}
				      \end{align*}
			      \end{minipage}\hfill
			      \begin{minipage}{0.5\textwidth}
				      \textbf{Sekundärseite:}\\ \ \\
				      \begin{align*}
					      Q_{sec} & = S_{sec} \cdot \sin(\varphi_{M.sec})   \\
					      P_{sec} & = 4,23\ \text{kVA} \cdot \sin(34^\circ) \\
					      Q_{sec} & = 2,37\ \text{kvar}
				      \end{align*}
			      \end{minipage}
		      }
	      \end{tcolorbox}

	      \textbf{Leistungsfaktor}\\ \ \\
	      \begin{tcolorbox}[colback=gray!30,
			      colframe=black,
			      width=0.9\textwidth,
		      ]
		      \parbox{\textwidth}{
			      \begin{minipage}{0.5\textwidth}
				      \textbf{Primärseite:}\\ \ \\
				      \begin{align*}
					      \lambda_{pri} & = \cos{(\varphi)}   \\
					      \lambda_{pri} & = \cos (36,7^\circ) \\
					      \lambda_{pri} & = 0,8
				      \end{align*}
			      \end{minipage}\hfill
			      \begin{minipage}{0.5\textwidth}
				      \textbf{Sekundärseite:}\\ \ \\
				      \begin{align*}
					      \lambda_{sec} & = \cos{(\varphi)} \\
					      \lambda_{sec} & = \cos (34^\circ) \\
					      \lambda_{sec} & = 0,83
				      \end{align*}
			      \end{minipage}
		      }
	      \end{tcolorbox}
	      \pagebreak

	\item Berechnen Sie die Verlustleistung des Transformators sowie den Wirkungsgrad mit
	      den Ergebnissen aus 4.2(a).

	      \begin{minipage}[r]{0.5\linewidth}
		      \begin{align*}
			      S_{V} & = S_{pri} - S_{sec}         \\
			      S_{V} & = (4,53 - 4,23)\ \text{kVA} \\
			      S_{V} & = 300\ VA                   \\
			      P_{V} & = P_{pri} - P_{sec}         \\
			      P_{V} & = (3,63 - 3,51)\ kW         \\
			      P_{V} & = 120\ W                    \\
			      Q_{V} & = Q_{pri} - Q_{sec}         \\
			      Q_{V} & = (2,71 - 2,27)\ \text{var} \\
			      Q_{V} & = 340\ \text{var}
		      \end{align*}
	      \end{minipage}
	      \begin{minipage}[l]{0.5\linewidth}
		      \begin{align*}
			      \eta_S & = \frac{S_{sec}}{S_{pri}} \\
			      \eta_S & = 0,93\ \widehat{=}\ 93\% \\
			      \eta_P & = \frac{P_{sec}}{P_{pri}} \\
			      \eta_P & = 0,97\ \widehat{=}\ 97\% \\
			      \eta_Q & = \frac{Q_{sec}}{Q_{pri}} \\
			      \eta_P & = 0,88\ \widehat{=}\ 88\% \\
		      \end{align*}
	      \end{minipage}

	\item Bestimmen Sie die Verlustleistung für 3.2(b) nach 2.1(d) und vergleichen Sie
	      das Ergebnis mit dem Wert aus 4.2(b). Begründen Sie Ihre Beobachtungen.\\ \ \\
	      \textbf{Eisen und Kupferverluste:}\\ \ \\
	      \begin{tcolorbox}[colback=gray!30,
			      colframe=black,
			      width=0.9\textwidth,
		      ]
		      \parbox{\textwidth}{

			      \begin{minipage}{0.5\textwidth}
				      \textbf{Eisenverluste:}\\ \ \\
				      \begin{align*}
					      P_{FE} & = P_0\ \left(\frac{U_{M.pri}}{U_N}\right)^2     \\
					      P_{FE} & = 36,2\ W\ \left(\frac{373\ V}{400\ V}\right)^2 \\
					      P_{FE} & = 31,591\ W
				      \end{align*}
			      \end{minipage}\hfill
			      \begin{minipage}{0.5\textwidth}
				      \textbf{Kupferverluste:}\\ \ \\
				      \begin{align*}
					      P_{Cu} & = P_K\ \left(\frac{I_{M.pri}}{I_N}\right)^2  \\
					      P_{Cu} & = 140\ W\ \left(\frac{7\ A}{7,2\ A}\right)^2 \\
					      P_{Cu} & = 132,330\ W
				      \end{align*}
			      \end{minipage}
		      }
	      \end{tcolorbox}
	      \begin{center}
		      \begin{minipage}[c]{0.5\linewidth}
			      \begin{align*}
				      \\
				      P_{V} & = P_{FE} \ +\ P_{Cu}                    \\
				      P_{V} & = 31,591\ W + 132,330\ W                \\
				      P_{V} & = 163,921\ W
				      \\
				      \\
				      \eta  & = 1-\frac{P_{V}}{S_{M.pri}}             \\
				      \eta  & = 1- \frac{132,330\ W }{4,53\text{ VA}} \\
				      \eta  & = 0,971\ \widehat{=}\ 97,1\ \%          \\
			      \end{align*}
		      \end{minipage}
	      \end{center}

	      Die Verlustleistung bewegt sich ungefähr in derselben Größenordnung, wobei
	      $P_{V.Mess}$ bei 120 W liegt und $P_{V.Theo}$ bei 163,921 W liegt. Es ist zu
	      beachten, dass der theoretische Wert etwas über dem gemessenen Wert liegt.\\
	      Die Diskrepanz zwischen dem theoretischen Wert ($P_{V.Theo} = 163,921\ W$) und
	      dem gemessenen Wert ($P_{V.Mess} = 120\ W$) könnte auf verschiedene Faktoren
	      zurückzuführen sein. Mögliche Gründe für einen höheren theoretischen Wert
	      könnten beispielsweise unberücksichtigte Verluste in der Messausrüstung, nicht
	      ideale Bedingungen während der Messung sein.

	\item Berechnen Sie aus den Messwerten nach 3.2(b) und dem Übersetzungsverhältnis ü
	      den relativen Spannungsfall $\Delta u'_2$!\\ \ \\

	      \begin{minipage}[r]{0.5\linewidth}
		      \begin{align*}
			      U_1 & = ü\cdot U_2            \\
			      ü   & = \frac{U_1}{U_2}       \\
			      ü   & = \frac{373\ V}{354\ V} \\
			      ü   & = 1,053                 \\
			      ü   & \approx 1
		      \end{align*}
	      \end{minipage}
	      \begin{minipage}[l]{0.5\linewidth}
		      \begin{align*}
			      I_2 & = ü\cdot I_1          \\
			      ü   & = \frac{I_2}{I_1}     \\
			      ü   & = \frac{6,9\ A}{7\ A} \\
			      ü   & = 0,985               \\
			      ü   & \approx 1
		      \end{align*}
	      \end{minipage}
	      \begin{align*}
		      \Delta u'_2 & = \frac{U_1\ -\ U'_2}{U_1}             \\
		      \Delta u'_2 & = \frac{373,67\ V - 354\ V}{373,67\ V} \\
		      \Delta u'_2 & = 0,053                                \\
	      \end{align*}
	\item Bestimmen Sie anhand der Formeln (11) - (13) den relativen Spannungsabfall für
	      3.2(b).

	      \begin{align*}
		      U_l         & = I_1(R_K\cos(\varphi_2)+X_K\sin(\varphi_2))                             \\
		      U_l         & = 7\ A\ (0,86\ \Omega\ \cos(34^\circ)\ +\ 0,38\ \Omega\ \sin(34^\circ))  \\
		      U_l         & = 6,478\ V                                                               \\
		      \\
		      U_q         & = I_1(R_K\cos(\varphi_2)+X_K\sin(\varphi_2))                             \\
		      U_q         & =  7\ A\ (0,86\ \Omega\ \sin(34^\circ)\ +\ 0,38\ \Omega\ \cos(34^\circ)) \\
		      U_l         & = 5,572 \ V                                                              \\                                                                \\
		      \\
		      \Delta u'_2 & = 1+\frac{U_l}{U_1} - \sqrt{1-\frac{U_q}{U_1}}                           \\
		      \Delta u'_2 & = 1+\frac{6,478\ V}{373,67\ V} - \sqrt{1-\frac{5,572 \ V}{373,67\ V}}    \\
		      \Delta u'_2 & = 0,017
	      \end{align*}
	      \pagebreak
	\item Vergleichen Sie die primär- und sekundärseitigen Spannungen und Ströme aus
	      3.2(b) und 3.2(c) miteinander.\\ \ \\

	      \subsection*{Vergleich der Eingangs- und Ausgangswerte:}

	      \textbf{Eingangsseite:}

	      \begin{itemize}
		      \item \textbf{Yy0-Verschaltung:} Yy0 hat eine höhere Eingangsspannung (\(373,667 \, V\) gegenüber \(197,667 \, V\))
		      \item \textbf{Dy5-Verschaltung:} Dy5 hat einen höheren Eingangsstrom (\(11,100 \, A\) gegenüber \(7 \, A\)).
	      \end{itemize}

	      \textbf{Ausgangsseite:}

	      \begin{itemize}
		      \item \textbf{Yy0-Verschaltung:} Yy0 hat eine höhere Ausgangsspannung (\(354,000 \, V\) gegenüber \(354 \, V\))
		      \item \textbf{Dy5-Verschaltung:} Dy5 hat einen niedrigeren Ausgangsstrom (\(6,300 \, A\) gegenüber \(6,900 \, A\)).

	      \end{itemize}

	      \begin{table}[h!]
		      \caption{Vergleich der primär- und sekundärseitigen Spannungen und Ströme}
		      \centering
		      \begin{tabular}{lcccc}
			      \hline
			                                    & \textbf{Spannung [V]} & \textbf{Strom [A]} & \textbf{Phasenwinkel [$^\circ$]} \\
			      \hline
			      \textbf{Yy0 Symmetrischelast} &                       &                    &                                  \\
			      Eingangsseite                 & $373,667$             & $7,000$            & $53,333$                         \\
			      Ausgangsseite                 & $354,000$             & $6,900$            & $-64,000$                        \\
			      \hline
			      \textbf{Dy5 Symmetrischelast} &                       &                    &                                  \\
			      Eingangsseite                 & $197,667$             & $11,100$           & $54,333$                         \\
			      Ausgangsseite                 & $354,000$             & $6,300$            & $-65,000$                        \\
			      \hline
		      \end{tabular}
	      \end{table}

\end{enumerate}
