\section{Aufgabenstellung }
\textbf{ACHTUNG:} Für den Schaltungsaufbau bitte nur Sicherheitsleitungen und Messgeräte mit Sicherheitsbuchsen verwenden! Die Gehäuse sind über Schutzleiter (grün/gelbe Leitung) an PE erden! Das Oszilloskop ist nur über Trennteiler bzw. Wandler anzuschließen! \\
\textbf{Der Transformator soll in der Baugruppe $Yy0$ geschaltet werden! Bitte überprüfen Sie das! \\
Nach jeder Messung ist die Versorgungsspannung des Transformators wieder auf 0 runterzufahren!}
\subsection{Messungen zur Ermittlung des vollständigen Ersatzschaltbildes des Drehstromtransformators}
Bitte nutzen Sie für den Aufbau 6 digitale Multimeter zur Spannungsmessung und 3 analoge Multimeter zur Strommessung. Für die Messung mit dem Oszilloskop verwenden Sie für die Spannung potentialfreie Tastköpfe und die Stromwandler zur Messung des Stromes. \\\ \\
\textbf{Leerlaufversuch: $(U_N \leq U_0 \leq 1,05 U_N)$}
\begin{enumerate}[label=\alph*)]
  \item Messen Sie die Leerlaufspannung $U_0$ sowie den Leerlaufstrom $I_0$ auf der Primärseite und die Leerlaufspannung $U_{20}$ auf der Sekundärseite für alle drei Phasen. Messen Sie ebenfalls alle zugehörigen Phasenwinkel der Primärseite. Beziehen Sie die Winkel auf $\underline U_{12}$ der Primärseite.
\end{enumerate}
\textbf{Kurzschlussversuch: ($I_k = I_N$, Spannung beachten!)}
\begin{enumerate}[label=\alph*)]
  \setcounter{enumi}{1}
  \item Bestimmen Sie den Nennstrom des Transformators. 
  \item Messen Sie bei primärseitigem Nennstrom (Kurzschlussstrom) die Kurzschlussspannung $U_K$ und den sekundären Kurzschlussstrom $I_2k$ aller drei Phasen sowie die zugehörigen Phasenwinkel. Beziehen Sie die Winkel auf der Primärseite auf $U_{12}$ der Primärseite. Auf der Sekundärseite messen Sie den Winkel zwischen $\underline U_{12}$ und $\underline I_1$ bezogen auf $\underline U_{12}$.  
\end{enumerate}

\subsection{Messungen bei symmetrischer Drehstromlast auf der Sekundärseite }
Belasten Sie den Transformator mit der ohmsch-induktiven Last nach Bild 11.\\
Bitte achten Sie auf den Nennstrom! ($U_{12} = 400\ V$ primärseitig). \\
Verwenden Sie zur Strommessung analoge und zur Spannungsmessung digitale Messgeräte.  
\begin{enumerate}[label=\alph*)]
  \item Dürfen Sie diese Last mit dem Transformator betreiben? Begründen Sie!
\end{enumerate}
\textbf{Schaltgruppe $Yy0$}
\begin{enumerate}[label=\alph*)]
  \setcounter{enumi}{1}
  \item Messen Sie alle Effektivwerte der Außenleiterspannungen und der Außenleiterströme auf der Primär- und Sekundärseite sowie alle zugehörigen Phasenwinkel. Beziehen Sie die Winkel auf der Primärseite auf $\underline U_{12}$ der Primärseite. Auf der Sekundärseite messen Sie den Winkel zwischen $\underline U_{12}$ und $\underline I_1$ bezogen auf $\underline U_{12}$. Messen Sie ebenfalls den Winkel zwischen den beiden Spannungen $\underline U_{12}$ auf der Primär- und der Sekundärseite. 
\end{enumerate}
\textbf{Schaltgruppe $Dy5$}
\begin{enumerate}[label=\alph*)]
  \setcounter{enumi}{2}
  \item Messen Sie alle Effektivwerte der Außenleiterspannungen und der Außenleiterströme auf der Primär- und Sekundärseite sowie alle zugehörigen Phasenwinkel. Beziehen Sie die Winkel auf der Primärseite auf $\underline U_{12}$ der Primärseite. Auf der Sekundärseite messen Sie den Winkel zwischen $\underline U_{12}$ und $\underline I_1$ bezogen auf $\underline U_{12}$. Messen Sie ebenfalls den Winkel zwischen den beiden Spannungen $\underline U_{12}$ auf der Primär- und der Sekundärseite. 
\end{enumerate}
\textbf{Die Primärseite bitte am Ende der Messung wieder in Stern verschalten! }
