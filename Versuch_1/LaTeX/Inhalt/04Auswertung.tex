\section{Auswertung}
\subsection{Dioden}
\begin{enumerate}[label=\alph*)]
	\item Stellen Sie die Kennlinien im linearen Maßstab dar, benutzen Sie dazu die korrigierten Messwerte (Spannungsfehlerschaltung beachten, Durchlassrichtung und Sperrrichtung unterschiedliche Maßstäbe, ggf. je zwei Diagramme).
	      % \newcommand\myeq{\stackrel{{{m}}}{=}}
	      % \begin{align*}
	      %   x &\myeq 10
	      % \end{align*}
	      \begin{figure}[h!]
		      \begin{center}
			      \includegraphics[width=0.85\textwidth]{img/4.1.a.1}
			      \caption{Silizium Diode in Durchlassrichtung}
		      \end{center}
	      \end{figure}

	      \begin{figure}[h!]
		      \begin{center}
			      \includegraphics[width=0.85\textwidth]{img/4.1.a.2}
			      \caption{Silizium Diode in Sperrrichtung}
		      \end{center}
	      \end{figure}

	      \pagebreak

	      \begin{figure}[h!]
		      \begin{center}
			      \includegraphics[width=0.85\textwidth]{img/4.1.a.3}
			      \caption{Germanium Diode in Durchlassrichtung}
		      \end{center}
	      \end{figure}

	      \begin{figure}[h!]
		      \begin{center}
			      \includegraphics[width=0.85\textwidth]{img/4.1.a.4}
			      \caption{Germanium Diode in Sperrrichtung}
		      \end{center}
	      \end{figure}

	      \pagebreak

	      \begin{figure}[h!]
		      \begin{center}
			      \includegraphics[width=0.85\textwidth]{img/4.1.a.5}
			      \caption{Z-Diode in Durchlassrichtung}
		      \end{center}
	      \end{figure}

	      \begin{figure}[h!]
		      \begin{center}
			      \includegraphics[width=0.85\textwidth]{img/4.1.a.6}
			      \caption{Z-Diode in Sperrrichtung}
		      \end{center}
	      \end{figure}

	      \pagebreak
	\item Berechnen Sie die Kennlinie der Si-Diode nach Gleichung 7, ermitteln Sie den Sperrstrom $I_s$ und den Emissionskoeffizienten m aus den Messungen. (Tipp: Den Emissionskoeffizienten bestimmen Sie, indem Sie zwei Messwerte in Gleichung 7 einsetzen und beide Gleichungen dividieren, so dass der Sperrstrom gekürzt werden kann. Diese Gleichung kann nun nach $m$ umgestellt werden.)
	      \begin{align*}
		      % I_1             & =I_s \cdot \left(e^\frac{U_1}{m\cdot U_T} - 1 \right)         \\
		      % I_2             & =I_s \cdot \left(e^\frac{U_2}{m\cdot U_T} - 1 \right)         \\
		      \frac{I_1}{I_2} & = \frac{I_s \cdot \left(e^\frac{U_1}{m\cdot U_T} - 1 \right)}
		      {I_s \cdot \left(e^\frac{U_2}{m\cdot U_T} - 1 \right)}                          \\
		      \frac{I_1}{I_2} & = \frac{e^\frac{U_1}{m\cdot U_T} - 1 }
		      {e^\frac{U_2}{m\cdot U_T} - 1}                                                  \\
	      \end{align*}
	      Die Rechnung lässt sich viel vereinfachen, wenn $\displaystyle{e^\frac{U}{m\cdot U_t} >> 1}$ ist. \\
	      Annahme: $m = m_{max} = 2$ und $10^3 >> 1$
	      \begin{align*}
		      e^\frac{U_{min}}{m_{max}\cdot U_t}         & \geq 10^3                                                    \\
		      \frac{U_{min}}{m_{max}\cdot U_t}           & \geq 3\cdot \ln(10)                                          \\
		      \frac{U_{min}}{2\cdot 25,3\cdot10^{-3}\ V} & \geq 3\cdot \ln(10)                                          \\
		      U_{min}                                    & \geq \frac{151,8\cdot \ln(10)}{10^3}\ V                      \\
		      U_{min}                                    & \geq 0,35\ V                                                 \\
		      \Rightarrow
		      \frac{I_1}{I_2}                            & = \frac{e^\frac{U_1}{m\cdot U_T}}
		      {e^\frac{U_2}{m\cdot U_T}}                                                                                \\
		      \frac{I_1}{I_2}                            & = e^\frac{U_1-U_2}{m\cdot U_T}                               \\
		      \ln\left(\frac{I_1}{I_2}\right)            & =  \frac{U_1-U_2}{m\cdot U_T}                                \\
		      m                                          & =  \frac{U_1-U_2}{\ln\left(\frac{I_1}{I_2}\right) \cdot U_T} \\
		      m                                          & =  \frac{0,4\ V - 0,89\ V}
		      {\ln\left(\frac{5\ \mu A}{0,5\ A}\right) \cdot 25\cdot 10^{-3}\ V} = 1,68                                 \\
		      I_s                                        & = \frac{I_1}{e^\frac{U_1}{m\cdot U_T} - 1}                   \\
		      I_s                                        & = \frac{5\ \mu A}{e^\frac{0,4\ V}
		      {1,68\cdot 25,3\cdot 10^{-3}\ V} - 1} = 4,029\cdot 10^{-10} A                                             \\
		      % I                                          & = 4,029\cdot 10^{-10}\ A\cdot
		      % \left(e^{\frac{U}{1,68\cdot 25,3\cdot 10^{-3}\ V}}-1\right)                                               \\
		      I                                          & = 4,029\cdot 10^{-10}\ A\cdot
		      \left(e^{\frac{U}{42,5\cdot 10^{-3}\ V}}-1\right)
	      \end{align*}
	      \pagebreak

	\item Stellen Sie die gemessene und berechnete Kennlinie der Si-Diode im halblogarithmischen Maßstab dar (berechnete Kennlinie als Linie, Messwerte als Punkte).

	      \begin{figure}[h!]
		      \begin{center}
			      \includegraphics[width=0.85\textwidth]{img/4.1.a.7}
			      \caption{Kennlinie der Si-Diode aus dem theoretischem Verlauf und der Messungen}
		      \end{center}
	      \end{figure}

	\item Beschriften Sie die fünf aufgenommenen Oszillogramme eindeutig.

	      \begin{figure}[h!]
		      \begin{center}
			      \includegraphics[width=0.85\textwidth]{img/V1/3.2.Widerstand.png}
			      \caption{Oszillogramm des ohmschen Widerstands}
		      \end{center}
	      \end{figure}

	      \pagebreak
	      \begin{figure}[h!]
		      \begin{center}
			      \includegraphics[width=0.85\textwidth]{img/V1/3.2.Diode_Si.png}
			      \caption{Oszillogramm der Si-Diode}
		      \end{center}
	      \end{figure}

	      \begin{figure}[h!]
		      \begin{center}
			      \includegraphics[width=0.85\textwidth]{img/V1/3.2.Diode_Ge.png}
			      \caption{Oszillogramm der Germanium-Diode}
		      \end{center}
	      \end{figure}

	      \pagebreak
	      \begin{figure}[h!]
		      \begin{center}
			      \includegraphics[width=0.85\textwidth]{img/V1/3.2.Diode_Ze.png}
			      \caption{Oszillogramm der Z-Diode}
		      \end{center}
	      \end{figure}

	      \begin{figure}[h!]
		      \begin{center}
			      \includegraphics[width=0.85\textwidth]{img/V1/3.2.Diode_Led.png}
			      \caption{Oszillogramm der Leuchtdiode}
		      \end{center}
	      \end{figure}
\end{enumerate}

\pagebreak
\subsection{Temperaturabhängiger Widerstand}
\begin{enumerate}[label=\alph*)]
	\item Ermitteln Sie für jeden Messwert den Widerstand (Spannungsfehlerschaltung beachten!)
	      \begin{table}[h!]
		      \caption{Temperaturabhängiger Widerstand: Strom- und Widerstandswerte}
		      \begin{center}
			      \begin{tabular}[c]{c|c|c}
				      \hline
				      \multicolumn{1}{c|}{\textbf{Temperatur in $^\circ C$}} &
				      \multicolumn{1}{c|}{\textbf{Strom in $mA$}}            &
				      \multicolumn{1}{c}{\textbf{Widerstand in $\Omega$}}                     \\
				      \hline
				      22,5                                                   & 2    & 1000,00 \\
				      25                                                     & 2,2  & 909,09  \\
				      30                                                     & 2,4  & 833,33  \\
				      35                                                     & 2,7  & 740,74  \\
				      40                                                     & 2,9  & 689,66  \\
				      45                                                     & 3,5  & 571,43  \\
				      50                                                     & 4    & 500,00  \\
				      55                                                     & 5    & 400,00  \\
				      60                                                     & 6    & 333,33  \\
				      65                                                     & 6,7  & 298,51  \\
				      70                                                     & 8,6  & 232,56  \\
				      75                                                     & 10   & 200,00  \\
				      80                                                     & 10,5 & 190,48  \\
				      85                                                     & 13   & 153,85  \\
				      90                                                     & 15   & 133,33  \\
				      95                                                     & 20   & 100,00  \\
				      100                                                    & 35   & 57,14   \\
				      \hline
			      \end{tabular}
		      \end{center}
	      \end{table}
	      \begin{figure}[h!]
		      \begin{center}
			      \includegraphics[width=0.95\textwidth]{img/4.2.a.1}
		      \end{center}
		      \caption{Strom-Temperatur-Kennlinie und Widerstand-Temperatur-Kennlinie}
	      \end{figure}
	      \pagebreak

	\item Berechnen Sie die Widerstandskennlinie näherungsweise nach Gleichung 8. Die Werkstoffkonstanten A und B sind aus den Messwerten bei minimaler und maximaler Temperatur zu ermitteln.
	      \begin{align*}
		      R(T)                                    & = A\cdot e^{\frac{B}{T}}                                                                \\
		      A                                       & = \frac{R(T)}{e^{\frac{B}{T}}}                                                          \\
		      \frac{R(T_1)}{R(T_2)}                   & = \frac{A}{A} \cdot \frac{e^{\frac{B}{T_1}}}{e^{\frac{B}{T_2}}}                         \\
		      \frac{R(T_1)}{R(T_2)}                   & = e^{\frac{B\cdot T_2-B\cdot T_1}{T_1\cdot T_2}}                                        \\
		      \ln \left(\frac{R(T_1)}{R(T_2)} \right) & = {\frac{B\cdot( T_2- T_1)}{T_1\cdot T_2}}                                              \\
		      B                                       & =\frac{\ln \left(\frac{R(T_1)}{R(T_2)} \right)\cdot {T_1\cdot T_2}}{( T_2- T_1)}        \\
		      B                                       & =\frac{\ln \left(\frac{1000\ \Omega}{57,14\ \Omega}\right)
		      \cdot {(295,65\ K \cdot 374,15\ K)}}{(374,15\ K - 295,65\ K)}                                                                     \\
		      B                                       & =4033\ K                                                                                \\
		      A                                       & = \frac{R(T_1)}{e^{\frac{B}{T_1}}} = \frac{1000\ \Omega}{e^{\frac{4033\ K}{295,65\ K}}} \\
		      A                                       & = 1,19\ m\Omega                                                                         \\
		      R(T)                                    & = 1,19\ m\Omega \cdot e^{\frac{4033\ K}{T}}                                             \\
	      \end{align*}
	\item Stellen Sie beide Kennlinien in einem Diagramm da (berechnete Kennlinie als Linie, Messwerte als Punkte).
	      \begin{figure}[h!]
		      \begin{center}
			      \includegraphics[width=0.7\textwidth]{img/4.2.c.1}
		      \end{center}
		      \caption{Widerstandskennlinie aus dem theoretischem Verlauf und der Messungen}\label{img:4.2.c.1}
	      \end{figure}

	      \pagebreak
	\item Vergleichen Sie die Kennlinien.

	      In der Abbildung \ref{img:4.2.c.1}  ist der Verlauf der Messdaten in Orange über dem theoretischen Verlauf in Blau dargestellt. Beide Verläufe, sowohl der theoretische als auch der gemessene, zeigen eine exponentiell abnehmende Charakteristik.

	      Die Abweichung zwischen den beiden Verläufen resultiert aus der Herausforderung der präzisen Temperaturmessung. Die Temperatur kann nicht exakt erfasst werden. Zusätzlich gestaltet sich die genaue Ablesung des Stroms als problematisch, da die Temperatur schnell abnimmt, was zu schnellen Änderungen in der Stromstärke führt.
\end{enumerate}


\pagebreak
\subsection{Transistor}
\begin{enumerate}[label=\alph*)]
	\item Stellen Sie die Eingangskennlinien nach 3.4.1 im linearen Maßstab in einem Diagramm dar.
	      \begin{figure}[h!]
		      \begin{center}
			      \includegraphics[width=0.85\textwidth]{img/4.3.a.1}
		      \end{center}
		      \caption{Eingangskennlinien des Transistors}
	      \end{figure}

	\item Stellen Sie das Ausgangskennlinienfeld nach 3.4.2 im linearen Maßstab in einem Diagramm dar.
	      \begin{figure}[h!]
		      \begin{center}
			      \includegraphics[width=0.85\textwidth]{img/4.3.b.1}
		      \end{center}
		      \caption{Ausgangskennlinienfeld des Transistors}
	      \end{figure}

	      \pagebreak
	\item Beschriften Sie das aufgenommene Oszillogramm eindeutig.
	      \begin{figure}[h!]
		      \begin{center}
			      \includegraphics[width=0.85\textwidth]{img/V1/Ausgang.png}
		      \end{center}
		      \caption{Oszillogramm der Ausgangskennlinienfelder}
	      \end{figure}
\end{enumerate}
